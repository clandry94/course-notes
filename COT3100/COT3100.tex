\documentclass[english, 11pt]{article}
\usepackage{../notes}
\usepackage{turnstile}
\usepackage{qtree}
\usepackage{flagderiv}
\usepackage{pdfpages}
\usepackage{scrextend}
\usepackage{mdframed}

% Uncomment these for a different family of fonts
% \usepackage{cmbright}
% \renewcommand{\sfdefault}{cmss}
% \renewcommand{\familydefault}{\sfdefault}
\newcommand{\thiscoursecode}{COT 3100}
\newcommand{\thiscoursename}{Applications of Discrete Structures}
\newcommand{\thisprof}{Dr. Alper Ungor}
\newcommand{\me}{Conor Landry}
\newcommand{\thisterm}{Spring 2016}
\newcommand{\website}{clandry94.github.io}

% Headers
\chead{\thiscoursename \ Course Notes}
\lhead{\thisterm}


%%%%%% TITLE %%%%%%
\newcommand{\notefront} {
\pagenumbering{roman}
\begin{center}
\vspace{0in}\includegraphics[scale=0.5]{../UF.png}

% {\ttfamily \url{\website}} {\small}

\textbf{\Huge{\noun{\thiscoursecode}}}{\Huge \par}

{\large{\noun{\thiscoursename}}}\\ \vspace{0.1in}

  {\noun \thisprof} \ $\bullet$ \ {\noun \thisterm} \ $\bullet$ \ {\noun {University of Florida}} \\

  \end{center}
  }


% Begin Document
\begin{document}
\setcounter{secnumdepth}{3}
  % Notes front
  \notefront
  % Table of Contents and List of Figures
  \tocandfigures
  % Abstract
  \doabstract{These notes are intended as a resource for myself; past, present, or future students of this course, and anyone interested in the material. The goal is to provide an end-to-end resource that covers all material discussed in the course displayed in an organized manner. If you spot any errors or would like to contribute, please contact me directly.}
\setcounter{secnumdepth}{3}
\section{Foundations of Logic: Overview}


\begin{addmargin}[1em]{2em} % 1em left, 2em right
  \begin{itemize}
      \item Propositional Logic
  	  \begin{itemize}
  		\item Basic Definitions
  		\item Equivalence Rules \& Derivations
  	  \end{itemize}
      \item Predicate Logic
  	  \begin{itemize}
  		\item Predicates
  		\item Quantified Predicate Expressions
  		\item Equivalences \& Derivations
  	  \end{itemize}
  \end{itemize}
\end{addmargin}

\subsection{Propositional Logic}
\begin{addmargin}[1em]{2em} % 1em left, 2em right
    \begin{mdframed}
    	\begin{defn}[Propositional Logic]\label{Propositional Logic}
    		The logic of compound statements built from simpler statements using \textit{Boolean Connectives}.
    	\end{defn}
    \end{mdframed}
    \medskip
    Applications:
    \begin{itemize}
      \item Design of digital electronic circuits
      \item Expressing conditions in computer programs
      \item Queries to databses \& search engines
    \end{itemize}
\end{addmargin}
\begin{addmargin}[2em]{2em}
  \subsubsection{Basic Definitions}
    \begin{addmargin}[1em]{2em} % 1em left, 2em right
        \begin{mdframed}
        	\begin{defn}[Proposition]\label{Propositional}
        		A \textbf{\textit{proposition (p, q, r, ...)}} is simply a \textit{statement i.e. a declarative sentence) with a definite meaning}, having
        		a \textit{truth value} that's either \textit{true} or \textit{false} (never both, neither, or somewhere in between).
          \end{defn}
        \end{mdframed}

        \medskip

        In \textit{probability theory}, we assign \textit{degrees of certainty} to propositions. For now we will
        just use True/False.

        \paragraph{Examples of Propositions}
          \begin{itemize}
            \item ``It is raining.''
          \end{itemize}
          \medskip
          Not propositions:
          \begin{itemize}
            \item ``Who's there?'' (interrogative, question)
          \end{itemize}

          \paragraph{Boolean Operators}
          \begin{itemize}
            \item Negation (NOT)
            \item Conjunction (AND)
            \item Disjunction (OR)
            \item Exclusive-Or (XOR)
            \item Implication (IF)
            \item Bi-conditional (IFF)
          \end{itemize}

          \paragraph{Operators \& Connectives}
          \begin{itemize}
            \item An \textit{operator} or \textit{connective} combines one or
            more \textit{operand} expressions into a larger expression
            (e.g. ``+'' in numeric expressions).
            \item \textit{Unary} operators take 1 operand (e.g. -3).
              \begin{itemize}
                \item The unary \textit{negation operator} ``\( \neg \)''
                (NOT) transforms a proposition into its logical \textit{negation}
                \begin{itemize}
                  \item Example: If p = ``I have brown hair.'', then
                  \( \neg \) p = ``I do not have brown hair.''
                    \begin{tabular}{c|cc}
                      p & $\neg$ p \\
                      \hline
                      T & F \\
                      F & T
                    \end{tabular}
                \end{itemize}
              \end{itemize}
            \item \textit{Binary} operators take 2 operands (e.g. 3 x 4).
            \item \textit{Propositional or Boolean} operators operate on
            propositions or truth values instead of on numbers.
          \end{itemize}


    \end{addmargin}


  \subsubsection{Equivalence Rules \& Derivations}


\end{addmargin}
\subsection{Predicate Logic}
\subsubsection{Predicates}
\subsubsection{Quantified Predicate Expressions}
\subsubsection{Equivalences \& Derivations}





  %\begin{align*}
    %3 & = 1 + 2 \\
      %& = 1 + 1 + 1
  %\end{align*}



  %The \nameref{addition} rule is very good.

 % \begin{lstlisting}[language=lisp]
  %(define sum (lambda args (foldr + 0 args)))
  %\end{lstlisting}

  %\tc{this is code}

  %%%%%%%%%%%%%%%%%%%%%%%%%%%%%%%%%%%%%%%%%%%
  \end{document}
