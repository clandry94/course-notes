\documentclass[english, 11pt]{article}
\usepackage{../notes}
\usepackage{turnstile}
\usepackage{qtree}
\usepackage{flagderiv}
\usepackage{pdfpages}


% Uncomment these for a different family of fonts
% \usepackage{cmbright}
% \renewcommand{\sfdefault}{cmss}
% \renewcommand{\familydefault}{\sfdefault}
\newcommand{\thiscoursecode}{COT 3100}
\newcommand{\thiscoursename}{Applications of Discrete Structures}
\newcommand{\thisprof}{Dr. Alper Ungor}
\newcommand{\me}{Conor Landry}
\newcommand{\thisterm}{Spring 2016}
\newcommand{\website}{clandry94.github.io}

% Headers
\chead{\thiscoursename \ Course Notes}
\lhead{\thisterm}


%%%%%% TITLE %%%%%%
\newcommand{\notefront} {
\pagenumbering{roman}
\begin{center}
\vspace{0in}\includegraphics[scale=0.5]{../UF.png}

% {\ttfamily \url{\website}} {\small}

\textbf{\Huge{\noun{\thiscoursecode}}}{\Huge \par}

{\large{\noun{\thiscoursename}}}\\ \vspace{0.1in}



  {\noun \thisprof} \ $\bullet$ \ {\noun \thisterm} \ $\bullet$ \ {\noun {University of Florida}} \\

  \end{center}
  }


% Begin Document
\begin{document}

  % Notes front
  \notefront
  % Table of Contents and List of Figures
  \tocandfigures
  % Abstract
  \doabstract{These notes are intended as a resource for myself; past, present, or future students of this course, and anyone interested in the material. The goal is to provide an end-to-end resource that covers all material discussed in the course displayed in an organized manner. If you spot any errors or would like to contribute, please contact me directly.}

\section{Foundations of Logic: Overview}



  \begin{itemize}
    \item Propositional Logic
	  \begin{itemize}
		\item Basic Definitions
		\item Equivalence Rules \& Derivations
	  \end{itemize}
    \item Predicate Logic
	  \begin{itemize}
		\item Predicates
		\item Quantified Predicate Expressions
		\item Equivalences \& Derivations
	  \end{itemize}
  \end{itemize}


\subsection{Propositional Logic}
	\begin{defn}[Propositional Logic]\label{Propositional Logic}
		The logic of compound statements built from simpler statements using \textit{Boolean Connectives}
	\end{defn}

\subsubsection{Basic Definitions}
	\begin{defn}[Propositional Logic]\label{Propositional Logic}
		A \textbf{\textit{proposition (p, q, r, ...)}} is simply a \textit{statement i.e. a declarative sentence) with a definite meaning}, having
		a \textit{truth value} that's either \textit{true} or \textit{false} (never both, neither, or somewhere in between)
	\end{defn}

\subsubsection{Equivalence Rules \& Derivations}
\subsection{Predicate Logic}
\subsubsection{Predicates}
\subsubsection{Quantified Predicate Expressions}
\subsubsection{Equivalences \& Derivations}





  %\begin{align*}
    %3 & = 1 + 2 \\
      %& = 1 + 1 + 1
  %\end{align*}

 

  %The \nameref{addition} rule is very good.

 % \begin{lstlisting}[language=lisp]
  %(define sum (lambda args (foldr + 0 args)))
  %\end{lstlisting}

  %\tc{this is code}

  %%%%%%%%%%%%%%%%%%%%%%%%%%%%%%%%%%%%%%%%%%%%%%%
  \end{document}
